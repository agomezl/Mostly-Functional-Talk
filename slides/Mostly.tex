\documentclass[rail]{beamer}
%\documentclass[handout]{beamer}
\usecolortheme[RGB={153,0,0}]{structure}
\usetheme{Rochester}
%\useinnertheme{rounded}
\setbeamertemplate{blocks}[rounded][shadow=true]
\useinnertheme{rectangles}
\usepackage[utf8]{inputenc}
\usepackage{listings}
\usepackage{natbib}
\usepackage{bibentry}
\usepackage{color}
\usepackage[spanish]{babel}
\usepackage{amssymb,latexsym}
\usepackage{xcolor}
\usepackage{hyperref}
\usepackage{alltt}

\setbeamertemplate{navigation symbols}{}
%\bibliography{Erlang}

\lstdefinestyle{style1}{
  belowcaptionskip=1\baselineskip,
  breaklines=true,
  xleftmargin=\parindent,
  showstringspaces=false,
  basicstyle=\footnotesize\ttfamily\color{blue},
  keywordstyle=\bfseries\color{green!40!black},
  commentstyle=\itshape\color{purple!40!black},
  identifierstyle=\color{black},
  stringstyle    =\color{red},
}
\lstset{style=style1,escapeinside=@@}%


\title{``Mostly functional'' programming}
\subtitle{does not work}
\author{Alejandro Gómez-Londoño}
\date{31th March, 2014}
\institute{EAFIT University}

\begin{document}
\begin{frame}
  \titlepage
\end{frame}

\begin{frame}{Introduction}
  \begin{quote}
    ``Conventional programming languages are large, complex, and
    inflexible. Their limited expressive power is inadequate to
    justify their size and cost.''\footnote[frame,1] {John
      Backus. 1978. Can programming be liberated from the von Neumann
      style?: a functional style and its algebra of
      programs. Commun. ACM 21, 8 (August 1978), 613-641}
  \end{quote}
\end{frame}

\begin{frame}[fragile]{Introduction}

  \begin{lstlisting}[language=C]
    void foo(int * arr, int len){
      int i;
      for(i=0;i<len;i++)
        arr[i] += 1;
    }
  \end{lstlisting}
  \pause
  \begin{lstlisting}[language=Haskell]
    foo :: (Num a) => [a] -> [a]
    foo arr = map (+1) arr
  \end{lstlisting}
\end{frame}

\begin{frame}{Imperative programming}
  \begin{quote}
    ``\dots is a programming paradigm that describes computation in terms
    of statements that change a program state''\footnote[frame,1]
    {Wikipedia contributors, ``Imperative programming''.
      Wikipedia, The Free Encyclopedia (Accessed June 16, 2014)}
  \end{quote}
  \pause
  \begin{itemize}[<+->]
  \item There is a global state.
  \item Variables $\approx$ storage cells.
  \item Assignments statements $\approx$ fetching and storing.
  \item Control statements $\approx$ Jump and test instructions.
  \end{itemize}
\end{frame}

\begin{frame}{Imperative programming}{The problem}
  \begin{quote}
    ``In a parallel/concurrent/distributed world, however, a single
    global state is an unacceptable bottleneck. so the foundational
    assumption of imperative programming that underpins most
    contemporary programming languages is starting to crumble''
  \end{quote}

\end{frame}


\begin{frame}{Functional programming}
  \begin{quote}
    ``\dots is a style of programming which models computations as the
    evaluation of expressions''\footnote[frame,1]
    {HaskellWiki contributors, ``Functional programming'',
      HaskellWiki, Haskell.org (Accessed June 16, 2014)}
  \end{quote}
  \begin{itemize}[<+->]
  \item Higher-order functions
  \item Immutable data
  \item Referential transparency
  \item Side effects through monads
  \item Lazy evaluation
  \end{itemize}
\end{frame}

\begin{frame}{Side effects}
  \begin{quote}
    ``... a function or expression is said to have a side effect if, in
    addition to returning a value, it also modifies some state or has
    an observable interaction with calling functions or the outside
    world.''\footnote[frame,1]
    {Wikipedia contributors, ``Side effect (computer science)''.
      Wikipedia, The Free Encyclopedia (Accessed June 16, 2014)}
  \end{quote}
  \pause
  \begin{itemize}[<+->]
  \item Modify global variables
  \item Write/Read a file
  \item Thread/Network communication
  \item IO actions in general
  \end{itemize}

\end{frame}

\begin{frame}{The problem}
  \begin{quote}
    ``There is a trend in the software industry to sell `mostly
    functional' programming as the silver bullet for solving problems
    developers face with concurrency, parallelism (manycore), and, of
    course, Big Data.''
    \footnote[frame,1] {Erik Meijer. 2014. The curse of the
      excluded middle. Commun. ACM 57, 6 (June 2014),50-55.}
  \end{quote}
  \pause
  \begin{itemize}[<+->]
  \item MapReduce
  \item Callbacks
  \item Deferred execution
  \end{itemize}
\end{frame}

\begin{frame}{The problem}
    \begin{quote}
    ``Just like `mostly secure', `mostly pure' is wishful
    thinking. The slightest implicit imperative effect erases all the
    benefits of purity, just as a single bacterium can infect a
    sterile wound''
    \footnote[frame,1] {Erik Meijer. 2014. The curse of the
      excluded middle. Commun. ACM 57, 6 (June 2014),50-55.}
  \end{quote}
\end{frame}


\begin{frame}[fragile]{The problem}{Deferred execution}
  \begin{lstlisting}[language=C]
    static bool LT30(int x) {
      Console.Write("{0}? < 30\n", x);
      return x < 30;
    }
    static bool MT20(int x) {
      Console.Write("{0}? > 20\n", x);
      return x > 20;
    }

    var q0 = new[]{ 1, 25, 40, 5, 23 }.Where(LT30);
    var q1 = q0.Where(MT20);

    foreach (var r in q1){
      Console.WriteLine("[{0}]\n",r);
    }
  \end{lstlisting}
\end{frame}

\begin{frame}[fragile]{The problem}{Deferred execution (output)}
\begin{verbatim}
     1? < 30
     1? > 20
     25? < 30
     25? > 20
     25
     40? < 30
     5? < 30
     5? > 20
     23? < 30
     23 > 20
     23
\end{verbatim}
\end{frame}

\begin{frame}[fragile]{The problem}{Exceptions and Laziness}
  \begin{lstlisting}[language=C]
    var xs = new[]{ 9, 8, 7, 6, 5, 4, 3, 2, 1, 0 };
    IEnumerable<int> q;

    try   { q = xs.Select(x=>1/x); }
    catch { q = new int[]; }

    foreach(var z in q){
      Console.WriteLine(z): // throws here
    }
  \end{lstlisting}
\end{frame}

\begin{frame}[fragile]{Down the Rabbit Hole}{Optimizations}
  \begin{lstlisting}[language=C]
    string Ha() {
      var ha = "Ha";
      Console.Write(ha);
      return ha;
    }

    // prints HaHa
    var haha = Ha()+Ha();

    // prints Ha
    var ha = Ha();
    var haha = ha+ha;
  \end{lstlisting}
\end{frame}

\begin{frame}[fragile]{Down the Rabbit Hole}{Abolish state mutation is not enough}
  \begin{lstlisting}[language=erlang]
    new_cell(X) -> spawn(fun() -> cell(X) end).

    cell(Val) ->
      receive
        {set, NewVal} -> cell(NewVal);
        {get, Pid}    -> Pid!{return, Val}, cell(Val);
        {dispose}     -> {}
      end.

    set_cell(Cell, NewVal) -> Cell!{set, NewVal}.
    get_cell(Cell) -> Cell!{get, self()},
      receive
         {return, Val} -> Val
      end.
    dispose_cell(Cell) -> Cell!{dispose}.

  \end{lstlisting}
\end{frame}

\begin{frame}{Down the Rabbit Hole}{Summary}
  \begin{itemize}[<+->]
  \item Functional features can be tricky when mixed with imperative programs.
  \item Imperative programs have side effects are EVERYWHERE.
  \item Abolish some side effects it's not enough.
  \item A program without side effects is useless.
  \end{itemize}
\end{frame}

\begin{frame}[fragile]{Fundamentalist Functional Programming}{All is (not) lost}
  \begin{lstlisting}[language=C]
    int foo(int a, int b);
  \end{lstlisting}
  \pause
  \hfill\\
  \begin{lstlisting}[language=Haskell]
    foo :: Int -> Int -> Int
  \end{lstlisting}
  \pause
  \hfill\\
  \begin{lstlisting}[language=Haskell]
    foo :: Int -> Int -> IO Int
  \end{lstlisting}
\end{frame}

\begin{frame}{Fundamentalist Functional Programming}
  \begin{quote}
    ``To understand how fundamentalist functional programming might
    help solve the concurrency problem, it is important to understand
    that it is not just imperative programming without side effects,
    which, as we have seen, is useless''
    \footnote[frame,1] {Erik Meijer. 2014. The curse of the
      excluded middle. Commun. ACM 57, 6 (June 2014),50-55.}
  \end{quote}
\end{frame}

\begin{frame}[fragile]{Informal Introduction to Monads}

  \begin{quote}
    Monads are a way of chaining computations that usually carry some
    effect
  \end{quote}
  \pause
  \begin{lstlisting}[language=Haskell]

    -- The injection function (return)
    return :: a -> m a
    @\pause@
    -- infix application function (bind)
    (>>=)  :: m a -> (a -> m b) -> m b

  \end{lstlisting}


\end{frame}

\begin{frame}[fragile]{Informal Introduction to Monads}
  \begin{lstlisting}[language=Haskell]
    class Monad m where
      (>>=)  :: m a -> (a -> m b) -> m b
      return :: a -> m a

    fooA :: a -> M b

    fooB :: b -> M c
  \end{lstlisting}
\end{frame}

\begin{frame}{Informal Introduction to Monads}{Usefull monads}
  \pause
  \begin{itemize}[<+->]
  \item \texttt{Maybe}
  \item \texttt{[]}
  \item \texttt{Either e}
  \item \texttt{ST}
  \item \texttt{STM}
  \end{itemize}
\end{frame}

\begin{frame}[fragile]{Informal Introduction to Monads}{The IO Monad}
  \begin{lstlisting}
    readFile :: FilePath -> IO String
    @\pause@
    writeFile :: FilePath -> String -> IO ()
    @\pause@
    getArgs :: IO [String]
    @\pause@
    forkIO :: IO () -> IO ThreadId
    @\pause@
    forkProcess :: IO () -> IO ProcessID
    @\pause@
    getLine :: IO String
    @\pause@
    putStrLn :: String -> IO ()
  \end{lstlisting}
\end{frame}
\end{document}
